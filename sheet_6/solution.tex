
\documentclass[12pt]{article}
 
\usepackage[margin=1in]{geometry} 
\usepackage{amsmath,amsthm,amssymb,scrextend}
\usepackage{fancyhdr}
\pagestyle{fancy}
\DeclareMathOperator{\rng}{Rng}
\DeclareMathOperator{\dom}{Dom}
\newcommand{\R}{\mathbb R}
\newcommand{\cont}{\subseteq}
\newcommand{\N}{\mathbb N}
\newcommand{\Z}{\mathbb Z}
\usepackage{tikz}
\usepackage{pgfplots}
\usepackage{amsmath}
\usepackage[mathscr]{euscript}
\let\euscr\mathscr \let\mathscr\relax% just so we can load this and rsfs
\usepackage[scr]{rsfso}
\usepackage{amsthm}
\usepackage{amssymb}
\usepackage{multicol}
%\usepackage{ngerman}
\usepackage[colorlinks=true, pdfstartview=FitV, linkcolor=blue,
citecolor=blue, urlcolor=blue]{hyperref}

\DeclareMathOperator{\arcsec}{arcsec}
\DeclareMathOperator{\arccot}{arccot}
\DeclareMathOperator{\arccsc}{arccsc}
\newcommand{\ddx}{\frac{d}{dx}}
\newcommand{\dfdx}{\frac{df}{dx}}
\newcommand{\ddxp}[1]{\frac{d}{dx}\left( #1 \right)}
\newcommand{\dydx}{\frac{dy}{dx}}
\let\ds\displaystyle
\newcommand{\intx}[1]{\int #1 \, dx}
\newcommand{\intt}[1]{\int #1 \, dt}
\newcommand{\defint}[3]{\int_{#1}^{#2} #3 \, dx}
\newcommand{\imp}{\Rightarrow}
\newcommand{\un}{\cup}
\newcommand{\inter}{\cap}
\newcommand{\ps}{\mathscr{P}}
\newcommand{\set}[1]{\left\{ #1 \right\}}
\newtheorem*{sol}{Solution}
\newtheorem*{claim}{Claim}
\newtheorem{problem}{Problem}
\begin{document}
 

\lhead{Formal Logic Sheet 6}
\chead{Robert Feldhans, Sebastian Mueller}
\rhead{\today}


\section*{Exercise 21: (Not a law)}

\subsection*{a}

$U_F = \{z\}$, $P^F=\{(x | x \in U_F)\}$\\
$U_G = \{y\}$, $P^G=\{(x | x \in U_G)\}$\\
The only case in which we found this true was if the U's of F and G differ and $\forall x(F \lor G)$ uses a union of both $U_F$ and $U_G$. 

\subsection*{b}

$U_F = \N$, $P^F=\{(x | x \in \N, x=3)\}$\\
$U_G = \N$, $P^G=\{(x | x \in \N, x=4)\}$\\
Both structures hold individually, but not when combined.

\section*{Exercise 22: (Equivalence vs consequence)}

\section*{Exercise 23: (Blue-eyed vampires)}

We model our this Problem as follows: As convention, we will give persons truth-values, just like atomic formulae, where elfs will be true (”1”) and vampires will be false (”0”). Similarily, we will give eye colors truth values, where blue eyes will be true (”1”) and brown eyes will be false (”0”).\\
Every person C has an eye color E and a parent P. Thus:\\
$F_1= (\lnot E \land \lnot P) \Rightarrow \lnot C$\\
Furthermore:\\
$F_2= P \Rightarrow E$\\
And lastly:\\
$F_3=\lnot E \Rightarrow \lnot C$

\section*{Exercise 24: (Skolem normal form)}

%ANARCHY
\end{document}