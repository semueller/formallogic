
\documentclass[12pt]{article}
 
\usepackage[margin=1in]{geometry} 
\usepackage{amsmath,amsthm,amssymb,scrextend}
\usepackage{fancyhdr}
\pagestyle{fancy}
\DeclareMathOperator{\rng}{Rng}
\DeclareMathOperator{\dom}{Dom}
\newcommand{\R}{\mathbb R}
\newcommand{\cont}{\subseteq}
\newcommand{\N}{\mathbb N}
\newcommand{\Z}{\mathbb Z}
\usepackage{tikz}
\usepackage{pgfplots}
\usepackage{amsmath}
\usepackage[mathscr]{euscript}
\let\euscr\mathscr \let\mathscr\relax% just so we can load this and rsfs
\usepackage[scr]{rsfso}
\usepackage{amsthm}
\usepackage{amssymb}
\usepackage{multicol}
%\usepackage{ngerman}
\usepackage[colorlinks=true, pdfstartview=FitV, linkcolor=blue,
citecolor=blue, urlcolor=blue]{hyperref}

\DeclareMathOperator{\arcsec}{arcsec}
\DeclareMathOperator{\arccot}{arccot}
\DeclareMathOperator{\arccsc}{arccsc}
\newcommand{\ddx}{\frac{d}{dx}}
\newcommand{\dfdx}{\frac{df}{dx}}
\newcommand{\ddxp}[1]{\frac{d}{dx}\left( #1 \right)}
\newcommand{\dydx}{\frac{dy}{dx}}
\let\ds\displaystyle
\newcommand{\intx}[1]{\int #1 \, dx}
\newcommand{\intt}[1]{\int #1 \, dt}
\newcommand{\defint}[3]{\int_{#1}^{#2} #3 \, dx}
\newcommand{\imp}{\Rightarrow}
\newcommand{\un}{\cup}
\newcommand{\inter}{\cap}
\newcommand{\ps}{\mathscr{P}}
\newcommand{\set}[1]{\left\{ #1 \right\}}
\newtheorem*{sol}{Solution}
\newtheorem*{claim}{Claim}
\newtheorem{problem}{Problem}
\begin{document}
 

\lhead{Formal Logic Sheet X}
\chead{Robert Feldhans, Sebastian Mueller}
\rhead{\today}


\section*{Exercise 5:  (Laws of Logic)}

\subsection*{1.}
$F \land (F \lor G) \equiv F$\\
$\Leftrightarrow (F \land F) \lor (F \land G) \equiv F$ \hspace*{2cm} see part 2\\
$\Leftrightarrow F \lor (F \land G) \equiv F$\\\\
So the truth table for both formulae looks like this:\\\\
\begin{tabular}{  l | l | l }
	F G & $F \lor G$ & $F \land (F \lor G) $  \\ \hline
	0 0 & 0 & 0  \\
	0 1 & 1 & 0  \\
	1 0 & 1 & 1  \\
	1 1 & 1 & 1  \\
\end{tabular}

\subsection*{2.}
$F \land (G \lor H) \equiv (F \land G) \lor (F \land H)$\\
\begin{tabular}{  l | l | l | l | l | l}
	F G H & $G \lor H$ & $F \land (G \lor H) $ & $F \land G$ & $F \land H$ & $(F \land G) \lor (F \land H)$ \\ \hline
	0 0 0 & 0 & 0 & 0 & 0 & 0\\
	0 0 1 & 1 & 0 & 0 & 0 & 0\\
	0 1 0 & 1 & 0 & 0 & 0 & 0\\
	0 1 1 & 1 & 0 & 0 & 0 & 0\\
	1 0 0 & 0 & 0 & 0 & 0 & 0\\
	1 0 1 & 1 & 1 & 0 & 1 & 1\\
	1 1 0 & 1 & 1 & 1 & 0 & 1\\
	1 1 1 & 1 & 1 & 1 & 1 & 1\\
\end{tabular}\\\\\\
$F \lor (G \land H) \equiv (F \lor G) \land (F \lor H)$\\
\begin{tabular}{  l | l | l | l | l | l}
	F G H & $G \land H$ & $F \lor (G \land H) $ & $F \lor G$ & $F \lor H$ & $(F \lor G) \land (F \lor H)$ \\ \hline
	0 0 0 & 0 & 0 & 0 & 0 & 0\\
	0 0 1 & 0 & 0 & 0 & 1 & 0\\
	0 1 0 & 0 & 0 & 1 & 0 & 0\\
	0 1 1 & 1 & 1 & 1 & 1 & 1\\
	1 0 0 & 0 & 1 & 1 & 1 & 1\\
	1 0 1 & 0 & 1 & 1 & 1 & 1\\
	1 1 0 & 0 & 1 & 1 & 1 & 1\\
	1 1 1 & 1 & 1 & 1 & 1 & 1\\
\end{tabular}

\subsection*{3.}
$\neg (F \land G) \equiv \neg F \lor \neg G$\\
\begin{tabular}{  l | l | l }
	F G & $F \land G$ & $\neg F \lor \neg G$  \\ \hline
	0 0 & 0 & 1  \\
	0 1 & 0 & 1  \\
	1 0 & 0 & 1  \\
	1 1 & 1 & 0  \\
\end{tabular}\\\\
$\neg (F \lor G) \equiv \neg F \land \neg G$\\
\begin{tabular}{  l | l | l }
	F G & $F \lor G$ & $\neg F \land \neg G$  \\ \hline
	0 0 & 0 & 1 \\
	0 1 & 1 & 0  \\
	1 0 & 1 & 0  \\
	1 1 & 1 & 0  \\
\end{tabular}

\section*{Exercise 6: (Two proofs)}

\section*{Exercise 7: (CNF and DNF)}

\section*{Exercise 8: (Switch and and or)}

%ANARCHY
\end{document}