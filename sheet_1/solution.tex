
\documentclass[12pt]{article}
 
\usepackage[margin=1in]{geometry} 
\usepackage{amsmath,amsthm,amssymb,scrextend}
\usepackage{fancyhdr}
\pagestyle{fancy}
\DeclareMathOperator{\rng}{Rng}
\DeclareMathOperator{\dom}{Dom}
\newcommand{\R}{\mathbb R}
\newcommand{\cont}{\subseteq}
\newcommand{\N}{\mathbb N}
\newcommand{\Z}{\mathbb Z}
\usepackage{tikz}
\usepackage{pgfplots}
\usepackage{amsmath}
\usepackage[mathscr]{euscript}
\let\euscr\mathscr \let\mathscr\relax% just so we can load this and rsfs
\usepackage[scr]{rsfso}
\usepackage{amsthm}
\usepackage{amssymb}
\usepackage{multicol}
%\usepackage{ngerman}
\usepackage[colorlinks=true, pdfstartview=FitV, linkcolor=blue,
citecolor=blue, urlcolor=blue]{hyperref}

\DeclareMathOperator{\arcsec}{arcsec}
\DeclareMathOperator{\arccot}{arccot}
\DeclareMathOperator{\arccsc}{arccsc}
\newcommand{\ddx}{\frac{d}{dx}}
\newcommand{\dfdx}{\frac{df}{dx}}
\newcommand{\ddxp}[1]{\frac{d}{dx}\left( #1 \right)}
\newcommand{\dydx}{\frac{dy}{dx}}
\let\ds\displaystyle
\newcommand{\intx}[1]{\int #1 \, dx}
\newcommand{\intt}[1]{\int #1 \, dt}
\newcommand{\defint}[3]{\int_{#1}^{#2} #3 \, dx}
\newcommand{\imp}{\Rightarrow}
\newcommand{\un}{\cup}
\newcommand{\inter}{\cap}
\newcommand{\ps}{\mathscr{P}}
\newcommand{\set}[1]{\left\{ #1 \right\}}
\newtheorem*{sol}{Solution}
\newtheorem*{claim}{Claim}
\newtheorem{problem}{Problem}
\begin{document}
 

\lhead{Formal Logic Sheet 1}
\chead{Robert Feldhans, Sebastian Mueller}
\rhead{\today}


\section*{Exercise 1: (Elves and Vampires)}

Leading Thoughts: We know, that any given person A either a vampire, always lying, or an elf, always telling the truth, is. As convention, we will give persons truth-values, just like atomic formulae, where elfs will be true (''1'') and vampires will be false (''0'').\\
Any statement F given by person A is either true and A is an elf, or it is false and A is an vampire. Thus, we can establish a new ''meta''-statement G, which must always be true: $G = (A \land F) \oplus ( \neg A \land \neg F )$. 

\subsection*{1.A}

We model F as follows:\\
$F = \neg A \lor \neg B$ \\
So the truth table looks like this:\\\\
\begin{tabular}{  l | l | l | l | l}
	A B & $F$ & $A \land F $ & $\neg A \land \neg F$ & $G$ \\ \hline
	0 0 & 1 & 0 & 0 & 0 \\
	0 1 & 1 & 0 & 0 & 0 \\
	1 0 & 1 & 1 & 0 & 1 \\
	1 1 & 0 & 0 & 0 & 0 \\
\end{tabular} \\\\
Thus our meta-statement G holds only true for $A=1$ (elf), $B=0$ (vampire).

\subsection*{1.B}

We model F as follows:\\
$F = \neg A \oplus B$ \\
So the truth table looks like this:\\\\
\begin{tabular}{  l | l | l | l | l}
	A B & $F$ & $A \land F $ & $\neg A \land \neg F$ & $G$ \\ \hline
	0 0 & 1 & 0 & 0 & 0 \\
	0 1 & 0 & 0 & 1 & 1 \\
	1 0 & 0 & 0 & 0 & 0 \\
	1 1 & 1 & 1 & 0 & 1 \\
\end{tabular} \\\\
Thus our meta-statement G holds true for $B=1$ (elf). A is ambig, meaning it could either be an elf or a vampire.

\subsection*{1.C}

This is a special case of 1.B, where we could just assume $B=0$. Our meta-statement G is not satisfiable in this case, meaning the statement cannot be made by elfs or vampires.

\subsection*{1.D}

\subsection*{1.E}

\subsection*{1.F}

We model F as follows:\\
$F = \neg A \land B$ \\
So the truth table looks like this:\\\\
\begin{tabular}{  l | l | l | l | l}
	A B & $F$ & $A \land F $ & $\neg A \land \neg F$ & $G$ \\ \hline
	0 0 & 0 & 0 & 1 & 1 \\
	0 1 & 1 & 0 & 0 & 0 \\
	1 0 & 0 & 0 & 0 & 0 \\
	1 1 & 0 & 0 & 0 & 0 \\
\end{tabular} \\\\
Thus our meta-statement G holds true for $A=B=0$ (vampire).


\section*{Exercise 2: (Borromean formulas)}

Choose \\
$ 
A_1 = A \land B \\
A_2 = B \land C \\
A_3 = \neg C \lor \neg A 
$ \\ \\
- $
	A_1 \land A_2 = A \land B \land B \land C = A \land B \land C $ 
	satisfiable for A = B = C = 1 \\
- $	A_1 \land A_3 = (A \land B) \land (\neg C \lor \neg A) = B \land ( (A \land \neg C) \lor \underset{unsatisfiable}{(A \land \neg A)) } = B \land A \land \neg C
	$ \\ satisfiable for A = B = 1, C = 0 \\
- $A_2 \land A_3 =$ similar to $A_1 \land A2
$, now choose B = C = 1, A = 0 to satisfy formula \\
- $A_1 \land A_2 \land A_3 = A \land B \land B \land C \land (\neg C \lor \neg A) = A \land B \land (\underset{unsatisfiable}{(\neg C \land C)}\lor(\neg A \land \neg C)) = \underset{unsatisfiable}{A \land \neg A} \land B \land C = 0
$ , for all possible inputs \\

\section*{Exercise 3: (XOR)}

\begin{tabular}{  l | l | l }
	A B & $ A \oplus B = 1 $ iff $A \neq B$ & $ (A \lor B) \land \neg(A \lor B) $ \\ \hline
	0 0 & 0 & 0 \\
	0 1 & 1 & 1 \\
	1 0 & 1 & 1 \\
	1 1 & 0 & 0 \\
\end{tabular} 
\\

The truth table shows that both formulas evaluate to the same value for all possible inputs, hence they are equivalent.

\section*{Exercise 4: (Truth value tables)}
$
F_1 =  \neg (A \implies B) = \neg (\neg A \lor B) = A \land \neg B   \\ \\
 F_2 = \neg (\neg A \lor \neg (\neg B \lor \neg A)) = A \land (\neg B \lor \neg A) = (A \land \neg B) \lor \underset{= unsatisfiable}{(A \land \neg A)} = A \land \neg B \\
F_3 = (A \land B) \land( \neg B \lor C) = \underset{unsatisfiable}{(A \land B \land \neg B)} \lor (A \land B \land C) = A \land B \land C \\
F_4 = A \Leftrightarrow (B \Leftrightarrow C) \\
$ \\
Evaluated in a truth table: \\ \\
\begin{tabular}{  l | l | l | l | l}
	C A B & $F_1$ & $F_2$ & $F_3$ & $F_4$ \\ 
	\hline
	0 0 0 & 0 & 0 & 0 & 1 \\
	0 0 1 & 0 & 0 & 0 & 0 \\
	0 1 0 & 1 & 1 & 0 & 0 \\
	0 1 1 & 0 & 0 & 0 & 0 \\
	1 0 0 & - & - & 0 & 0 \\
	1 0 1 & - & - & 0 & 0 \\
	1 1 0 & - & - & 0 & 0 \\
	1 1 1 & - & - & 1 & 1 \\
\end{tabular}
\\

From the lecture we know that two Formulas are equivalent if their truth tables are equivalent, hence $F_1$ and $F_2$ are equivalent, $F_3$ and $F_4$ are not, $F_1$ and $F_3$ aren't either.
%comment
%ANARCHY
\end{document}