
\documentclass[12pt]{article}
 
\usepackage[margin=1in]{geometry} 
\usepackage{amsmath,amsthm,amssymb,scrextend}
\usepackage{fancyhdr}
\pagestyle{fancy}
\DeclareMathOperator{\rng}{Rng}
\DeclareMathOperator{\dom}{Dom}
\newcommand{\R}{\mathbb R}
\newcommand{\cont}{\subseteq}
\newcommand{\N}{\mathbb N}
\newcommand{\Z}{\mathbb Z}
\usepackage{tikz}
\usepackage{pgfplots}
\usepackage{amsmath}
\usepackage[mathscr]{euscript}
\let\euscr\mathscr \let\mathscr\relax% just so we can load this and rsfs
\usepackage[scr]{rsfso}
\usepackage{amsthm}
\usepackage{amssymb}
\usepackage{multicol}
%\usepackage{ngerman}
\usepackage[colorlinks=true, pdfstartview=FitV, linkcolor=blue,
citecolor=blue, urlcolor=blue]{hyperref}

\DeclareMathOperator{\arcsec}{arcsec}
\DeclareMathOperator{\arccot}{arccot}
\DeclareMathOperator{\arccsc}{arccsc}
\newcommand{\ddx}{\frac{d}{dx}}
\newcommand{\dfdx}{\frac{df}{dx}}
\newcommand{\ddxp}[1]{\frac{d}{dx}\left( #1 \right)}
\newcommand{\dydx}{\frac{dy}{dx}}
\let\ds\displaystyle
\newcommand{\intx}[1]{\int #1 \, dx}
\newcommand{\intt}[1]{\int #1 \, dt}
\newcommand{\defint}[3]{\int_{#1}^{#2} #3 \, dx}
\newcommand{\imp}{\Rightarrow}
\newcommand{\un}{\cup}
\newcommand{\inter}{\cap}
\newcommand{\ps}{\mathscr{P}}
\newcommand{\set}[1]{\left\{ #1 \right\}}
\newtheorem*{sol}{Solution}
\newtheorem*{claim}{Claim}
\newtheorem{problem}{Problem}
\begin{document}
 

\lhead{Formal Logic Sheet 8}
\chead{Robert Feldhans, Sebastian Mueller}
\rhead{\today}


\section*{Exercise 29:  (Insane Vampires)}

\section*{Exercise 30: (Infinite Models)}
$ F = \forall x P(x,f(x)) \land \forall y \lnot P(y,y) \land \forall x \forall y \forall z ((P(x,y)\land P(y,z)) \Rightarrow P(x,z))$
\subsection*{a}
Find a model that satisfies F\\
$U^F = \Z\\
P^F(x,y) = \{ (x,y) | x < y\}\\
f(x): x \rightarrow x+1$

\section*{Exercise 31: (Undecidable Problem I)}
\subsection*{a}
\[
A =\begin{bmatrix}
0 1 0\\
  0  
\end{bmatrix}, 
B = \begin{bmatrix}
  1\\
1 1 0  
\end{bmatrix},
C = \begin{bmatrix}
1 0\\
0 1
\end{bmatrix}
\]
\[
B+C+A =\begin{bmatrix}
  1    1 0 0 1 0\\
1 1 0  0 1    0
\end{bmatrix}
\]

\subsection*{b}
\[
A =\begin{bmatrix}
0 1 0\\
 0 1  
\end{bmatrix}, 
B = \begin{bmatrix}
 1\\
1 0  
\end{bmatrix},
C = \begin{bmatrix}
 1 0\\
1 0 1
\end{bmatrix}
\]
In order to be able to finish a sequence one needs a piece where the shorter sequence matches the end of the longer one, eg in a) piece A has the shorter sequence "0" in the bottom row which matches the end of the upper sequence "010".\\
The pieces given here have no such ending piece, hence no finite sequence can be found.

\subsection*{c}
\[
A =\begin{bmatrix}
1\\
1 0  
\end{bmatrix}, 
B = \begin{bmatrix}
0\\
1 0  
\end{bmatrix},
C = \begin{bmatrix}
0 1 0\\
0 1
\end{bmatrix}
D = \begin{bmatrix}
1 1\\
1
\end{bmatrix}
\]
\[
D+B =\begin{bmatrix}
1 1  0\\
 1  1 0
\end{bmatrix}
\]


\subsection*{d}
\[
A =\begin{bmatrix}
1 0\\
1 0 1  
\end{bmatrix}, 
B = \begin{bmatrix}
011\\
1 1  
\end{bmatrix},
C = \begin{bmatrix}
1 0 1\\
0 1 1
\end{bmatrix}
\]
It is also not possible to find a sequence here for a similar argument as given in b). The only piece we can finish a sequence with is B. If B is the last piece, the piece before last would have to have a zero as the last symbol in the bottom row to match the zero in the first row of B. No such piece is provided.

\section*{Exercise 32: (Undecidable Problem II)}

For the sake of simplification, we will enumerate both set of tiles from 1 to 4, from the left to the right.

\subsection*{Left set of tiles}

Consider the following plane:\\
3 4 1 3\\
3 2 4 3\\
3 4 1 3\\
3 2 4 3\\
This plane can easily be infinitely extended in any direction.

\subsection*{Right set of tiles}

%ANARCHY
\end{document}