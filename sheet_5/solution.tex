
\documentclass[12pt]{article}
 
\usepackage[margin=1in]{geometry} 
\usepackage{amsmath,amsthm,amssymb,scrextend}
\usepackage{fancyhdr}
\pagestyle{fancy}
\DeclareMathOperator{\rng}{Rng}
\DeclareMathOperator{\dom}{Dom}
\newcommand{\R}{\mathbb R}
\newcommand{\cont}{\subseteq}
\newcommand{\N}{\mathbb N}
\newcommand{\Z}{\mathbb Z}
\usepackage{tikz}
\usepackage{pgfplots}
\usepackage{amsmath}
\usepackage[mathscr]{euscript}
\let\euscr\mathscr \let\mathscr\relax% just so we can load this and rsfs
\usepackage[scr]{rsfso}
\usepackage{amsthm}
\usepackage{amssymb}
\usepackage{multicol}
%\usepackage{ngerman}
\usepackage[colorlinks=true, pdfstartview=FitV, linkcolor=blue,
citecolor=blue, urlcolor=blue]{hyperref}

\DeclareMathOperator{\arcsec}{arcsec}
\DeclareMathOperator{\arccot}{arccot}
\DeclareMathOperator{\arccsc}{arccsc}
\newcommand{\ddx}{\frac{d}{dx}}
\newcommand{\dfdx}{\frac{df}{dx}}
\newcommand{\ddxp}[1]{\frac{d}{dx}\left( #1 \right)}
\newcommand{\dydx}{\frac{dy}{dx}}
\let\ds\displaystyle
\newcommand{\intx}[1]{\int #1 \, dx}
\newcommand{\intt}[1]{\int #1 \, dt}
\newcommand{\defint}[3]{\int_{#1}^{#2} #3 \, dx}
\newcommand{\imp}{\Rightarrow}
\newcommand{\un}{\cup}
\newcommand{\inter}{\cap}
\newcommand{\ps}{\mathscr{P}}
\newcommand{\set}[1]{\left\{ #1 \right\}}
\newtheorem*{sol}{Solution}
\newtheorem*{claim}{Claim}
\newtheorem{problem}{Problem}
\begin{document}
 

\lhead{Formal Logic Sheet 5}
\chead{Robert Feldhans, Sebastian Mueller}
\rhead{\today}


\section*{Exercise 17: (Models?)}

\subsection*{a}

Choose for example $x=3$, $y=5$, $z=4$\\
$\Rightarrow$ $x<z$, $z<y$, $x<z$, $\lnot (z < x)$\\
This holds true, so this is a model for F.

\subsection*{b}
$P=\{(n,n+1)| n \in \N \} \\
(x,y) = (x,x+1) \Rightarrow (y=x+1), \\
(z,y)=(z,x+1) \Rightarrow z=x, \\
(x,z)=(x,x+1) \Rightarrow (z=x+1)\\
(z=x)\land(z=x+1)$ unsatisfiable\\
So this is no model of F.



\subsection*{c}

$x=y'$, $z=y'$, $x=z'$, $\lnot(z=x')$\\
Derivatives are surjectiv, thus $x=z$. Furthermore either\\
$x=z' \land x'=z$ or\\
$\lnot(x'=z) \land \lnot(x=z')$.\\
In conclusion, this cannot be a model for F.

\subsection*{d}

Choose for example $y = \{1,2,3\}$, $z=\{1,2\}$, $x=\{1\}$\\
$\Rightarrow$ $x \subseteq y$, $z \subseteq y$, $x \subseteq z$, $\lnot (z \subseteq x)$\\
This holds true, so this is a model for F.

\section*{Exercise 18: (Models and non-models)}
$F=\forall x \exists y P(x,y,f(z))$\\
\subsection*{Terms}
$x,y,z,f(z)$\\
\subsection*{Partial Formulas}
$F,\exists y P(x,y,f(z)), \forall x P(x,y,f(z)), P(x,y,f(z)) $

\subsection*{Matrix}
$P(x,y,f(z))$

\subsection*{Structures that are not model for F:}

$U_A = \N$, $P^A=\{(x,y,z | x,y,z \in \N, y<x)\}$ $(x=1)$\\
$U_A = \R$, $P^A=\{(x,y,z | x,y,z \in \R, \pi x^2 = y^2)\}$ (radius of circle/sidelength of square which have the same area)

\subsection*{Structures that are model for F:}

$U_A = \Z$, $P^A=\{(x,y,z | x,y,z \in \Z, y<x)\}$\\
$U_A = \R$, $P^A=\{(x,y,z | x,y,z \in \R, y^2<x^2, y \neq x)\}$\\

\section*{Exercise 19: (Reflexive, symmetric, transitive)}

\section*{Exercise 20: (Small universes)}

%ANARCHY
\end{document}