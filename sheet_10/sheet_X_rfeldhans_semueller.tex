
\documentclass[12pt]{article}
 
\usepackage[margin=1in]{geometry} 
\usepackage{amsmath,amsthm,amssymb,scrextend}
\usepackage{fancyhdr}
\pagestyle{fancy}
\DeclareMathOperator{\rng}{Rng}
\DeclareMathOperator{\dom}{Dom}
\newcommand{\R}{\mathbb R}
\newcommand{\cont}{\subseteq}
\newcommand{\N}{\mathbb N}
\newcommand{\Z}{\mathbb Z}
\usepackage{tikz}
\usepackage{pgfplots}
\usepackage{amsmath}
\usepackage[mathscr]{euscript}
\let\euscr\mathscr \let\mathscr\relax% just so we can load this and rsfs
\usepackage[scr]{rsfso}
\usepackage{amsthm}
\usepackage{amssymb}
\usepackage{multicol}
%\usepackage{ngerman}
\usepackage[colorlinks=true, pdfstartview=FitV, linkcolor=blue,
citecolor=blue, urlcolor=blue]{hyperref}

\DeclareMathOperator{\arcsec}{arcsec}
\DeclareMathOperator{\arccot}{arccot}
\DeclareMathOperator{\arccsc}{arccsc}
\newcommand{\ddx}{\frac{d}{dx}}
\newcommand{\dfdx}{\frac{df}{dx}}
\newcommand{\ddxp}[1]{\frac{d}{dx}\left( #1 \right)}
\newcommand{\dydx}{\frac{dy}{dx}}
\let\ds\displaystyle
\newcommand{\intx}[1]{\int #1 \, dx}
\newcommand{\intt}[1]{\int #1 \, dt}
\newcommand{\defint}[3]{\int_{#1}^{#2} #3 \, dx}
\newcommand{\imp}{\Rightarrow}
\newcommand{\un}{\cup}
\newcommand{\inter}{\cap}
\newcommand{\ps}{\mathscr{P}}
\newcommand{\set}[1]{\left\{ #1 \right\}}
\newtheorem*{sol}{Solution}
\newtheorem*{claim}{Claim}
\newtheorem{problem}{Problem}
\begin{document}
 

\lhead{Formal Logic Sheet X}
\chead{Robert Feldhans, Sebastian Mueller}
\rhead{\today}


\section*{Exercise 37:  (Violate the Peano axioms!)}

Standard model:
$U_A = \N _0$\\
$S(x) = x+1$\\
$f(x,y) = x+y$\\
$g(x,y) = x*y$\\
Change model such that...\\

a)\\
Satisfy all except $\forall x 0 \ne S(x)$:\\
$U_A = \N_0 \cup \{-1\}$\\

b) \\
Satisfy all except $\forall x \forall y S(x) = S(y) \Rightarrow x = y$:\\
$U_A = \Z \setminus \{-1, 1\}$\\
$S(n) = |n| + 1$\\

c)\\ % das hier duerfte falsch sein
Satisfy all except $\forall x \forall y \ \ x*S(y) = f(g(x,y),x)$:\\
$U_A = \Z \setminus \{-1,1\}$\\
$f(x,y) = x + g(-1,y)$

\section*{Exercise 38: (Consequences of Peano axioms)}

\section*{Exercise 39: (Football experts)}

%Is it sufficient if we construct formulas or do we also need to give a structure to "prove" our formulas?
We construct a the structure $A=\{W, R, \alpha \}$ where\\ 
$W = \{1, 2, ... , 34\}$ (days of play in Bundesliga this season. Current day of play is 11)\\
$R$ the relation $<$ from ex. 40, here being interpretable as a timeline relation as $n<m =$ "n happened before m"\\ \\

$\alpha : \{M, B, D\} \times W \rightarrow \{0,1\}$\\
Results from matches on 22nd July 2017: $\alpha (M,1) = 0,\ \alpha (D,1) = 0,\ \alpha (B,1) = 0$\\
$\alpha(B,n) =
	\begin{cases}
	1 &, n \geq 12 \\
	1 &, \text{if Bremen won}, n < 12 \\
	0 & , \text{otherwise(lost)}
	\end{cases} \\
\alpha(M,n) =
	\begin{cases}
		1 &, \text{Munich won on day n} \\
		0 & , \text{otherwise(lost)}
	\end{cases} \\
\alpha(d,n) =
	\begin{cases}
	1 &, n > 12 \\
	1 &, \text{if Dortmund won}, n < 12 \\
	0 & , \text{otherwise(lost)}
	\end{cases} \\
$\\ \\
Atomic Formulas: M = Munich wins, B = Bremen wins, D = Dortmund wins\\ \\
a) B\\
b) $\lnot D$\\
c) $F = \diamond \lnot M \Rightarrow \lnot B$\\
d) $\diamond M \Rightarrow D$\\
e) $M \lor B \lor D$ \\ \\
At least the last statement is wrong. On July 22nd 2017, Bayern lost against Mailand, Dortmund against Bochum and Bremen against St. Pauli. This violates statement e. \\
%wtf ich wette sogar das Dirk darauf hinaus wollte. Ich finde die Fussball echt langweilig.

\section*{Exercise 40: (Frames)}

%Neither have we defined a way to evaluate multiple operators with a map, nor can I find any rule how to reduce $\diamond \diamond \diamond B$

%ANARCHY
\end{document}