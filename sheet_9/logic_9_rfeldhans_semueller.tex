
\documentclass[12pt]{article}
 
\usepackage[margin=1in]{geometry} 
\usepackage{amsmath,amsthm,amssymb,scrextend}
\usepackage{fancyhdr}
\pagestyle{fancy}
\DeclareMathOperator{\rng}{Rng}
\DeclareMathOperator{\dom}{Dom}
\newcommand{\R}{\mathbb R}
\newcommand{\cont}{\subseteq}
\newcommand{\N}{\mathbb N}
\newcommand{\Z}{\mathbb Z}
\usepackage{tikz}
\usepackage{pgfplots}
\usepackage{amsmath}
\usepackage[mathscr]{euscript}
\let\euscr\mathscr \let\mathscr\relax% just so we can load this and rsfs
\usepackage[scr]{rsfso}
\usepackage{amsthm}
\usepackage{amssymb}
\usepackage{multicol}
%\usepackage{ngerman}
\usepackage[colorlinks=true, pdfstartview=FitV, linkcolor=blue,
citecolor=blue, urlcolor=blue]{hyperref}

\DeclareMathOperator{\arcsec}{arcsec}
\DeclareMathOperator{\arccot}{arccot}
\DeclareMathOperator{\arccsc}{arccsc}
\newcommand{\ddx}{\frac{d}{dx}}
\newcommand{\dfdx}{\frac{df}{dx}}
\newcommand{\ddxp}[1]{\frac{d}{dx}\left( #1 \right)}
\newcommand{\dydx}{\frac{dy}{dx}}
\let\ds\displaystyle
\newcommand{\intx}[1]{\int #1 \, dx}
\newcommand{\intt}[1]{\int #1 \, dt}
\newcommand{\defint}[3]{\int_{#1}^{#2} #3 \, dx}
\newcommand{\imp}{\Rightarrow}
\newcommand{\un}{\cup}
\newcommand{\inter}{\cap}
\newcommand{\ps}{\mathscr{P}}
\newcommand{\set}[1]{\left\{ #1 \right\}}
\newtheorem*{sol}{Solution}
\newtheorem*{claim}{Claim}
\newtheorem{problem}{Problem}
\begin{document}
 

\lhead{Formal Logic Sheet 9}
\chead{Robert Feldhans, Sebastian Mueller}
\rhead{\today}


\section*{Exercise 33: (Post correspondence problem light)}

\section*{Exercise 34: (Decidable first-order logic)}
\subsection*{a}
\subsection*{b}

\section*{Exercise 35: (Undecidable problem III: Mortal Matrices)}
\subsection*{a}
This set of matrices is not a set of mortal matrices.\\
$A_1$ is just $-1 \cdot \mathbb{E}$, as well as $A_2^2$, $A_3 \cdot A_2$ and $A_3^2$, so multiplakation with it will only result in the zero matrix if is multiplied with the zero matrix. Similarly $A_2 \cdot A_3$ is just $\mathbb{E}$. Furthermore, $A_3$ is just $A_1 \cdot A_2$

\subsection*{b}
This set of matrices is a set of mortal matrices.\\

$ B_1 \cdot B_3 =
\begin{pmatrix}
2 & 2 \\
2 & 2 
\end{pmatrix}\\
$

$
B_3 \cdot B_2 = 
\begin{pmatrix}
0 & 2 \\
-1 & 0 
\end{pmatrix}\\
$

$B_1 \cdot 
\begin{pmatrix}
0 & 2 \\
-1 & 0 
\end{pmatrix} = 
\begin{pmatrix}
-2 & 2 \\
-2 & 2 
\end{pmatrix}\\
$ 

$
\begin{pmatrix}
2 & 2 \\
2 & 2 
\end{pmatrix} \cdot
\begin{pmatrix}
-2 & 2 \\
-2 & 2 
\end{pmatrix} =
\begin{pmatrix}
0 & 0 \\
0 & 0 
\end{pmatrix}\\
$

\subsection*{c}




\section*{Exercise 36: (More Mortal Matrices)}

We wrote a small python program (see attachments of the mail) to brute force our way to a solution, which was quite clever considering the solution is quite long:\\
$A \cdot B \cdot B \cdot A \cdot B \cdot A \cdot A \cdot B \cdot B \cdot B \cdot B \cdot A \cdot A \cdot A \cdot B \cdot B \cdot A = Zeromatrix$


%ANARCHY
\end{document}